\begin{thebibliography}{99}

\bibitem{babar_y4260} B.~Aubert {\it et al.} (BaBar Collaboration),
Phys. Rev. Lett. {\bf 95}, 142001 (2005).

\bibitem{cleo_y4260} Q.~He {\it et al.} (CLEO Collaboration),
Phys. Rev. D {\bf 74}, 091104(R) (2006).

\bibitem{belle_y4260} C.Z.~Yuan {\it et al.} (Belle Collaboration),
Phys. Rev. Lett. {\bf 99}, 182004 (2005).

\bibitem{galina}
G.~Pakhlova {\it et al.} (Belle Collaboration),
Phys. Rev. Lett. {\bf 98}, 092001 (2007);
Phys. Rev. Lett. {\bf 100}, 062001 (2008);
Phys. Rev. D {\bf 77}, 011103 (2008);
Phys. Rev. Lett. {\bf 101}, 172001 (2008); and
Phys. Rev. D {\bf 80}, 091101 (2009).

%\bibitem{bes2_R} J.Z.~Bai {\it et al.} (BESII Collaboration),
%Phys. Rev. Lett. {\bf 88}, 101802 (2002).

\bibitem{mo} X.H.~Mo {\it et al.},
Phys. Lett. {\bf B640}, 182 (2006).

\bibitem{pdg} J.~Beringer {\it et al.} (Particle Data Group),
 Phys. Rev. D {\bf 86}, 010001 (2012).

\bibitem{belle_pipiyns} K.-F.~Chen {\it et al.} (Belle Collaboration),
Phys. Rev. Lett. {\bf 100}, 112001 (2008).

\bibitem{belle_z_b} A.~Bondar {\it et al.} (Belle Collaboration),
Phys. Rev. Lett. {\bf 108}, 122001 (2012).

%\bibitem{conj} The inclusion of charge-conjugate modes
%is implied unless explicitly stated otherwise.

\bibitem{belle_zb_bbstr} I.~Adachi {\it et al.} (Belle Collaboration),
arXiv:1209.6450v2 [hep-exp].

\bibitem{voloshin}
A.E.~Bondar {\it et al.}, Phys. Rev. D {\bf 84}, 054010 (2011),
D.V.~Bugg, Europhys. Lett. {\bf 96}, 11002 (2011), I.V.~Danilkin,
V.D.~Orlovsky and Yu.A.~Simonov, Phys. Rev. D {\bf 85}, 034012 (2012),
C.-Y.~Cui, Y.-L.~Liu and M.-Q.~Huang, Phys. Rev. D {\bf 85}, 054014 (2012),
T.~Guo, L.~Cao, M.-Z.~Zhou and H.~Chen, arXiv:1106.2284 [hep-ph], and
J.-R.~Zhang, M.~Zhong and M.-Q.~Huang Phys. Lett. {\bf B704}, 312 (2011).

\bibitem{molecule}See, for example,
M.B.~Voloshin and L.B.~Okun,
JETP Lett. {\bf 23}, 333 (1976);
M. Bander, G.L.~Shaw and P.~Thomas,
Phys. Rev. Lett. {\bf 36}, 695 (1977);
A.~De~Rujula, H.~Georgi and S.L.~Glashow,
Phys. Rev. Lett. {\bf 38}, 317 (1977);
A.V. Manohar and M.B. Wise, Nucl. Phys. B {\bf 339}, 17 (1993);
N.A.~T\"{o}rnqvist, hep-ph/0308277 (2003);
F.E.~Close and P.R.~Page, Phys. Lett. B {\bf 578}, 119 (2003);
C.-Y.~Wong, Phys. Rev. C {\bf 69}, 055202 (2004);
S.~Pakvasa and M.~Suzuki, Phys. Lett. B {\bf 579}, 67 (2004);
E.~Braaten and M.~Kusunoki, Phys. Rev. D {\bf 69}, 114012 (2004);
E.S.~Swanson, Phys. Lett. B {\bf 588}, 189 (2004);
D.~Gamermann and E.~Oset, Phys. Rev. D {\bf 80}, 014003 (2009)
\& Phys. Rev. D {\bf 81}, 014029 (2010).

\bibitem{bes3_z3900} M.~Ablikim {\it et al.} (BESIII Collaboration),
Phys. Rev. Lett. {\bf 110}, 252001 (2013).

\bibitem{belle_z3900}
Z.Q.~Liu {\it et al.} (Belle Collaboration), Phys. Rev. Lett. {\bf 110}, 252002 (2013).

\bibitem{zhao} Q.~Wang, C.~Hanhart and Q.~Zhao, Phys. Rev. Lett. {\bf 111}, 132003 (2013).

\bibitem{mahajan}
N.~Mahajan, arXiv:1304.1301 [hep-ph],
M.B.~Voloshin, Phys.Rev. D {\bf 87}, 091501 (2013),
J.-R.~Zhang, Phys. Rev. D {\bf 87}, 116004 (2013),
F.-K. Guo, {\it et al.}, Phys. Rev. D {\bf 88}, 054007 (2013) and
C.-Y.~Cui {\it et al.}, arXiv:1304.1850.

\bibitem{bes3_z4025} M.~Ablikim {\it et al.} (BESIII Collaboration), arXiv:1308.2760 [hep-ex]
and M.~Ablikim {\it et al.} (BESIII Collaboration), arXiv:1309.1896 [hep-ex].

\bibitem{faccini}
R.~Faccini {\it et al.}, Phys. Rev. D {\bf 87}, 111102R (2013);
and M.~Karliner and S.~Nussinov, JHEP {\bf 1307}, 153 (2013).
See, also, A.~Ali, C.~Hambrock and W.~Wang, Phys. Rev. D {\bf 85},
054011 (2012).



\bibitem{BESIII} M.~Ablikim {\it et al.} (BESIII Collaboration),
Nucl. Istrum. and Methods Phys. Res., Sect. A {\bf 614}, 345 (2010).

\bibitem{evtgen} D.J.~Lange,
Nucl. Istrum. and Methods Phys. Res., Sect. A {\bf 462}, 152 (2001).

\bibitem{kkmc} S.~Jadach, B.F.L.~Ward, and Z.~Was,
Comput. Phys.  Commun. {\bf 130}, 260 (2000);
Phys. Rev. D {\bf 63}, 113009 (2001).

\bibitem{geant4}
S.~Agostinelli~{\etal} (Geant4 Collaboration),
Nucl. Istrum. and Methods Phys. Res., Sect. A {\bf 506}, 250 (2003).

\bibitem{boost} Z.Y.~Deng, {\it et al.},
High Energy Phys. Nucl. Phys. {\bf 30} 371, (2006).

\bibitem{lund} R.G.~Ping {\it et al.}, Chinese Phys. C {\bf 32}, 599 (2008).

\bibitem{pythia} T.~Sj\"{o}strand, S.~Mrenna and P.~Skands,
JHEP {\bf 026}, 0605 (2006).

\bibitem{recoil} We minimize the effect of the $D$ mass resolution by
plotting $M^{\rm recoil}=RM(\pi D)+M(D)-m_D$, where $RM(\pi D)$
is the recoil mass inferred from four-momentum conservation and $M(D)$ is the
measured $D$ mass.

%\bibitem{nonpeaking} Monte Carlo studies of potential background processes do not show
%any peaking behavior in $M(DD*bar)$.

\bibitem{psi4040_2_etajpsi} M.~Ablikim {\it et al.} (BESIII Collaboration),
Phys. Rev. D {\bf 86}, 071101 (2012).

%\bibitem{pole} The pole positions are $m_{\zc}=3882.3\pm 1.8$ ($3884.1\pm 1.8$) MeV and
%$\Gamma_{\zc}=24.6\pm 4.1$ ($26.6\pm 3.9$) MeV for the $\pip D^0$ ($\pim D^+$) tag modes.

\bibitem{pole} See, for example, A.R.~Bohm and N.L.~Harshman, arXiv:hep-ph/0001206 and
references cited therein.

\bibitem{factor} We use a second-order QED calculation and unpublished
BESIII energy-dependent measurements of $\sigma(\ee\rt\pi D\bar{D^*})$
to compute the radiative correction; see E. A. Kuraev
and V. S. Fadin, Yad. Fiz. {\bf 41}, 733 (1985)
[Sov. J. Nucl. Phys. {\bf 41}, 466 (1985)].

\end{thebibliography}
